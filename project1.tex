\documentclass{article}
\usepackage{graphicx}
\usepackage[paper=a4paper,margin=1in]{geometry}
\usepackage[fontsize=12pt]{fontsize}

\title{Math 475 -- Project 1 \\ Bobcat Population}
\author{Jackson Brienen}
\date{October 8 2025}

\begin{document}

\maketitle

\section{Introduction}
In this paper we discuss how we can model bobcat populations under a variety of environmental conditions and external manipulations (hunting and introduction). While there are a variety of modeling techniques that can be used for populations we focus on using a basic exponential model. With this come a variety of assumptions:

\begin{enumerate}
    \item There is no carrying capacity/ maximum population.
    \item The growth rate of the population is always constant, i.e. the growth rate in the first year will be the same as the hundredth year, and thousandth year, and so on.
    \item We assume, in most cases, that fractional populations are valid. While one half of a bobcat does not usually make sense, for the sake simplicity we will allow it to be the case unless we explicitly say we use integer based populations.
    \item All births and deaths happen at the end of a year/ start of a new year.
\end{enumerate}

This paper will be divided into three main topics. First, the analysis of the basic model under three different growth rates in both the long and short term. Second, we will how the model changes when hunting is introduced and devising a strategy to meet a stable population with hunting. Third and last, we will analyze how the model changes when external introduction is used along with devising a strategy to meet a stable population with introduction.

% \section{Methods} % maybe add this? I feel like the whole paper is going to describe the methods we use, so I don't think this is that needed.

\section{Basic Growth}

\section{Hunting}

\section{Population Introduction}

\end{document}